\documentclass[a4paper]{article}
\title{A Zero Cost Muon Ranger and Beam Monitor for the DUNE Near Detector}%for Overleaf

%% Language and font encodings
\usepackage[british]{babel} % set babel to british rather than american
\usepackage[utf8]{inputenc} % utf8 support in source code
\usepackage{lmodern}
\usepackage[T1]{fontenc} % better support for special characters in pdf but messes up title fonts, can be fixed by installing the debian package cm-super
\usepackage{textcomp}
\usepackage[svgnames]{xcolor}
\usepackage{placeins}

%% Sets page size and margins
\usepackage[a4paper,top=3cm,bottom=2cm,left=3cm,right=3cm,marginparwidth=1.75cm]{geometry}

\usepackage{lineno}
\linenumbers

%\usepackage{authblk} for nice, normal, author lists. we should be so lucky

\usepackage{hyperref}
\usepackage{mathtools} % math packages
\usepackage{amsfonts}
\usepackage{amssymb}
\usepackage{physics}
\usepackage[version=4]{mhchem}
\usepackage{tabu}
\usepackage{booktabs} % fancy tables

\usepackage{afterpage}

\usepackage{siunitx}
\sisetup{separate-uncertainty=true}
\DeclareSIUnit\radlen{\text{\ensuremath{X_{\mathrm{0}}}}}
\DeclareSIUnit\clight{\text{\ensuremath{c}}} % remove 0 subscript from speed of light

\usepackage[labelfont=bf]{caption} %make caption label bold

\usepackage{float}

\usepackage{graphicx}
\graphicspath{{./Figures/}}
\usepackage{microtype}   
\usepackage{soul}

\usepackage[autostyle]{csquotes} % recommended by biblatex
\usepackage{xpatch} % recommended by biblatex
\usepackage[backend=biber, giveninits, sorting=none, style=numeric-comp]{biblatex} % much more flexible than BibTeX
\addbibresource{bibliography.bib}

\newcommand*{\m}{\mathrm}
\newcommand{\comment}[1]{\textcolor{red}{[#1]}}
\newcommand\addcite{[{\color{blue} \underline{CITATION NEEDED}}]}

\begin{document}

\begin{center}
	
	{\Large \bf A Practical Muon Ranger for the DUNE Near Detector} 
	\vspace*{0.5cm}
	\setcounter{footnote}{0}  
	\def\A{\kern+.6ex\lower.42ex\hbox{$\scriptstyle \iota$}\kern-1.20ex a}
	\def\E{\kern+.5ex\lower.42ex\hbox{$\scriptstyle \iota$}\kern-1.10ex e}
	\small
	\newcommand{\Aname}[2]{#1}
	\def\titlefoot#1{\vspace{-0.01cm}\begin{center}{\bf #1}\end{center}}
	
	\Aname{J.~R.~Sinclair\footnote{Corresponding author: james.sinclair@lhep.unibe.ch}}{Bern},
	\Aname{M.~Weber}{Bern}, and 
	\Aname{C.~Wilkinson}{Bern}
	\titlefoot{Albert Einstein Center for Fundamental Physics, Laboratory for High Energy Physics, University of Bern, 3012 Bern, Switzerland\label{Bern}}
	
	\Aname{A.~Bross}{FNAL}
	\titlefoot{Fermi National Accelerator Laboratory, Batavia, IL 60510, USA\label{FNAL}}
	
\end{center}


The DUNE Near Detector (ND) construction will have to be staged. 
The final stage will see a LArTPC, ArgonCube, upstream of a magnetised GArTPC, the Multi-Purpose Detector (MPD).  
Stage zero has to provide the bare minimum for an oscillation analysis from day one of operation.

The oscillation analysis assumes ArgonCube from day one. 
ArgonCube will not change, beyond upgrades of internal components, for the duration of the DUNE ND operation.
The dimensions of ArgonCube are \SI{3}{\metre} tall, \SI{7}{\metre} wide, and \SI{5}{\metre} long in beam direction~\cite{LAr_size}.
These dimensions were optimised to contain hadronic showers and not forward-going muons. 
It is therefore necessary for ArgonCube to be coupled to a downstream muon spectrometer with good geometric acceptance, in order to contain muons.
The MPD will provide this in the final stage, but will not be included in stage zero.
%To maximise the fraction of muons from ArgonCube that penetrate into the spectrometer, it is also important to minimise uninstrumented material between the two active components.   

Here, we present a practical steel-scintillator muon ranger to operate downstream of ArgonCube at stage zero.   
The proposal is based on the MINOS-ND and the SBN Cosmic Ray Tagger (CRT), available at zero core cost.
If the MINOS-ND is already used in conjunction with the ArgonCube 2x2 demonstrator~\cite{2x2} in ProtoDUNE-ND, then the application to the DUNE ND would be straightforward.

Defining a minimum requirement for the stage zero spectrometer is not trivial. 
Studies have shown that neutrino energies \textgreater\SI{5}{\giga\electronvolt} do not contribute significantly to the most basic oscillation analysis)~\cite{elizabeth2}.%, as there are no oscillated \textgreater\SI{5}{\giga\electronvolt} $\nu_{\mathrm{e}}$ at the Far Detector (FD)~\cite{elizabeth2}. 
%However, that does not mean that \SI{8}{\giga\electronvolt} ND events do not inform our knowledge about modelling \SI{4}{\giga\electronvolt} FD events.
In the absence of a complete oscillation analysis, we consider basic properties of the flux, which diminishes after \SI{5}{\giga\electronvolt}, where the ND event rate falls to roughly 10\% of its value at the peak~\cite{LBL}.
Allowing some contingency, we set a limit of \SI{6}{\giga\electronvolt} muon momentum that must be contained by the near detector in stage zero.
Going beyond \SI{6}{\giga\electronvolt} is obviously desirable, but this will likely run into spatial constraints of the ND hall in the absence of a magnet.  

                
To provide good geometric acceptance, we assume transverse dimensions close to those of the MPD: \SI{3.5}{\metre} tall and \SI{6}{\metre} wide.
The depth in beam direction is dictated by the stopping range of a \SI{6}{\giga\electronvolt} muon; a MIP deposits \SI{2}{\mega\electronvolt\per\centi\metre\squared\per\gram}, given the density of steel at \SI{7.8}{\gram\per\centi\metre\cubed}, that corresponds to \SI{1.56}{\giga\electronvolt\per\metre}.
Therefore, neglecting the stopping power of scintillator, at least \SI{4}{\metre} of steel are required in the beam direction.  
Of course, the geometry should be optimised through simulation.   
The steel can be broken into 40 layers, each \SI{10}{\centi\metre} deep with a transverse area of \SI{21}{\metre\squared}; to instrument every layer including the front face would require \SI{861}{\metre\squared} of X-Y scintillator planes.
It is not necessary to instrument every layer, a scheme like that used in MINOS~\cite{MINOSDetectors}, with finer granularity upstream, should be implemented to reduce the required instrumentation.
Again, this will need optimising through simulation, based on the desired resolution at various momenta.    


Bern has provided the SBN with \SI{626}{\metre\squared} of X-Y (\SI{1250}{\metre\squared} single layer) SiPM-instrumented scintillator planes~\cite{CRT}, a single plane is shown in Figure~\ref{fig:CRT}.
These planes, complete with readout, DAQ, and power supplies, are currently located at Fermilab, where they are being used as the CRTs of MicroBooNE~\cite{uBCRT} and SBND. 
Each (X or Y) plane is \SI{2}{\centi\metre} thick, with surface areas ranging from \SI[product-units=repeat]{1.8x4.5}{\metre} to \SI[product-units=repeat]{0.96x2.72}{\metre}.
Therefore, their stacking would need optimising for deployment in the muon ranger.
A single module has a timing resolution of \SI{3}{\nano\second} and a position resolution of \SI{1.8}{\centi\metre}~\cite{CRT,uBCRT}. 
After the SBN, the reuse of the CRT will have to be negotiated, but it will fit the timeline of the DUNE ND.  

\begin{figure}[H]
	\centering{
		\includegraphics[width=.4\columnwidth]{Figures/CRT2}
		\includegraphics[width=.4\columnwidth]{Figures/CRT1}	
	}
	\caption{Left: A Cosmic Ray Tagger (CRT) module during construction; 16 scintillator plastic strips are visible. Right: A \SI[product-units=repeat]{1.8x4.1}{\metre} production module during testing at Bern~\cite{CRT}.}
	\label{fig:CRT}
\end{figure}

A possible source of steel for the ranger is the MINOS-ND. 
The MINOS-ND~\cite{MINOS_NIM} is a magnetic spectrometer formed of \SI{1}{\kilo\tonne} of steel and plastic scintillator.
It is currently located in the NuMI beam line at Fermilab.  
A schematic of the MINOS-ND is shown in Figure~\ref{fig:minos}, it consists of 282 planes of \SI{2.54}{\centi\metre} thick steel,  \SI{7.16}{\metre} in total. 
A drawback of the MINOS-ND steel is its surface area, which is only \SI[product-units=repeat]{4.8x3}{\metre}, and not of optimal shape.
However, only 158 of the planes are required to achieve the \SI{4}{\metre} depth of steel. 
The remaining 124 planes contain sufficient steel to broaden the 158 planes to the required width. 
The MINOS steel is low carbon, which is not a requirement of the DUNE ND, therefore the steel could potentially be exchanged for sections of more suitable dimensions.

\begin{figure}[H]
	\centering
	\includegraphics[width=.97\textwidth]{Figures/minos.png}
	\caption{Left: Top view of the MINOS near detector, showing the calorimeter and muon spectrometer (not to scale). Right: transverse view of a near detector plane. The shaded area shows a partially instrumented active scintillator plane, the dashed line shows the boundary of the fiducial region. The dotted line denotes the outline of a fully instrumented scintillator plane. Reproduced from Figure~2 of Ref.~\cite{MINOSDetectors}.}
	\label{fig:minos}
\end{figure}

Using the components described, the total dimensions of the muon ranger can be estimated.  
The \SI{626}{\metre\squared} of scintillator planes corresponds to 29 layers, given a surface area of \SI{21}{\metre\squared} for each layer.  
Each instrumented layer requires an X and Y plane, combined at \SI{4}{\centi\metre}, plus a support structure also estimated at \SI{4}{\centi\metre}, giving a total of \SI{8}{\centi\metre}. 
The total depth of the scintillator planes is therefore \SI{2.32}{\metre}, combined with the steel gives over all dimensions of \SI[product-units=repeat]{3.5x6x6.32}{\metre}.        

After stage zero, the muon ranger could be rebuilt along the downstream wall of the ND hall to provide a large area  beam monitor. 

Such a muon ranger being adopted as stage zero for the DUNE ND, would make it beneficial to demonstrate the coupling of a LArTPC to a muon range stack before being deployed in the ND.
For example, the ArgonCube 2x2 demonstrator and the MINOS-ND, where the required software could be developed, and the experience gained would guarantee that the stage zero detector is already well understood at relevant energies. 
The MINOS-ND scintillator and readout could be replaced with the SBN CRTs when the steel is transferred into the DUNE ND hall.      

\printbibliography

\end{document}



  
